\documentclass{proc}
\usepackage{url}
\linespread{1.2}

\begin{document}

\title{Project Proposal: A Survey and Performance Comparisons of Techniques to Construct Restricted Computing Environments}

\author{Brett Jia \hspace{1em} Jennifer Bi}

\maketitle

\section*{Abstract}

In modern computer systems, users and researchers often need to run untrusted and potentially malicious code, yet would like to protect their systems from harm. This is often achieved by creating a restricted environment in which the untrusted code then executes safely. We seek to survey the current state of restricted computing technologies and present a comparison of various programs that provide this restricted computing environment.

\section*{1. Introduction}

Despite the best efforts of systems and hardware designers, a major weakness of modern day operating systems is the high level of trust given to user-level application code to allow programs to do what they are designed to do. A word processor is permitted to read and write files to the local filesystem, whereas a web browser is permitted to open sockets and make network connections. Without these capabilities, programs would be rendered useless, much to a computer user's dissatisfaction.
\newline\newline
While many software applications (such as word processors and web browsers) are now distributed through secure channels and feature cryptographic signatures from trusted entities, there are still instances where users want to run untrusted code, but still keep their local system safe from any malicious side effects. For example, web scripting languages such as JavaScript are present in a large majority of websites, which run the untrusted scripts immediately as a webpage loads. As users cannot be expected to disable JavaScript completely (thus rendering most websites unusable), browser designers are forced to design policies and mechanisms to isolate this untrusted code from the rest of the computer. In another case, a security researcher may want to run malware on a local computer for testing and study, but also seek to protect the computer from harm. Additionally, a system administrator may want to open up servers for public use, but limit the operations each user can perform to prevent or limit damage to the overall system.
\newline\newline
In each of these situations, the typical solution is to create a sort of restricted environment for unsafe programs to run. Creating such a restricted environment comes in many forms, from using tools and syscalls built into an operating system, to emulating the machine instructions of an application binary, to even using full-featured virtual machines. Selecting a tool from this wide buffet of options can be difficult, and may vary depending on the use case or software to be restricted. In our research, we seek to survey the current state of restricted computing technologies by discussing the different techniques employed, then present a comparison of various programs that provide this restricted computing environment in both objective, hard metrics as well as subjective, soft metrics.

\section*{2. Literature Review}

In preparation for our research, we sought out existing literature on existing application isolation techniques. We discovered a diverse set of techniques used to create isolated and restricted environments for running untrusted code. Our findings are listed below in this section.
\newline\newline
A prior survey of isolation systems and techniques from 2009 \cite{viswanathan2009isolation} categorizes isolation techniques into the following categories:
\begin{enumerate}
    \item Language-based isolation (e.g. type safety and compiler verification)
    \item Sandbox-based isolation (e.g. interposition between an application and the operating system's syscalls)
    \item Virtual machine-based isolation (e.g. Type I and Type II hypervisors)
    \item Kernel-based isolation (e.g. isolation between processes)
    \item Hardware-based isolation (e.g. hardware support for virtual memory addresses)
    \item Physical isolation (e.g. airgapped machines)
\end{enumerate}
The taxonomy of isolation techniques and a list of surveyed systems presented in the aforementioned paper provide a useful categorization of restricted computing environment techniques and useful comparisons of the taxonomic categories listed above. For the purposes of our research, we are mainly interested in sandbox-based and virtual machine-based isolation.
\newline\newline
Another survey of virtualization technologies for untrusted code execution from 2012 \cite{wen2012virtualization} notes that untrusted code is usually executed in one of four categories:
\begin{enumerate}
    \item Application-level sandboxing through programming language features, kernel interposition, or resource access control policies
    \item Simulating an operating system environment through virtualizing an operating system's runtime API
    \item Containerizing an execution environment through kernel-supported isolation of all resources required by a process
    \item Virtualizing hardware and isolating full operating systems through Type I and Type II virtual machine monitors
\end{enumerate}
This survey by Wen et al. presents various examples of each type of restricted computing environment and discusses their strengths and weaknesses, however fails to present quantitative metrics to compare the technologies listed. Indeed, the research presented by Viswanathan and Neuman \cite{viswanathan2009isolation} also fails to present any sort of benchmarking for the surveyed technologies.
\newline\newline
Although useful in providing categorization and terminology, we find these two surveys to be dated, as newer forms of restricted computing environments have been developed in recent years. Our own literature review presents the following list of technologies (some of which have been discussed in the prior two surveys), to be expanded upon in following subsections:
\begin{enumerate}
    \item FreeBSD Jails \cite{kamp2000jails}
    \item File Monitoring and Access Control (FMAC) \cite{prevelakis2001fmac}
    \item Vx32 \cite{ford2008vx32}
    \item Native Client \cite{yee2009native}
    \item Apiary \cite{potter2010apiary}
    \item MBOX \cite{kim2013mbox}
    \item System sandbox (for Microsoft Windows) \cite{vokorokos2015sandboxMSWIN}
\end{enumerate}

\subsection*{2.1 FreeBSD Jails}

The FreeBSD jail concept \cite{kamp2000jails} introduces a method of providing restricted computing environments through kernel-level partitioning of processes into ``jails.'' A process in ``jail'' sees a traditional FreeBSD environment with traditional resources (e.g. processes, file systems, networking), but restricts the privileged, root-level operations the process can perform (e.g. sending signals to processes outside of the jail). This functionality is implemented through the \texttt{jail} syscalls, requiring modifications to the FreeBSD kernel, and takes advantage of the functionality of \texttt{chroot}, with modifications, to protect the filesystem.

\subsection*{2.2 File Monitoring and Access Control (FMAC)}

Prevelakis and Spinellis \cite{prevelakis2001fmac} present a filesystem sandboxing solution by isolating a process' access to the filesystem through a File Monitoring and Access Control (FMAC) tool, which presents a mirror of the local filesystem. The sandbox mounts the tool as a filesystem and uses \texttt{chroot} to restrict a process to only view the isolated file system. Access control policies can be set up to allow the application to perform certain read or write operations on select files in the mirrored filesystem, effectively protecting the local filesystem from unauthorized access and modification.

\subsection*{2.3 Vx32}

Ford and Cox \cite{ford2008vx32} present a virtual machine for the 32-bit x86 instruction set, Vx32, as a library on which to build sandboxed applications. The virtual machine utilizes the x86 processor's segmentation hardware to sandbox guest data and employs dynamic translation to rewrite privileged instructions and sandbox guest code. Vx32 runs completely in user-mode and requires no modifications to the host kernel, although it has hardware, host kernel, and guest code requirements.

\subsection*{2.4 Native Client}

Google's Native Client \cite{yee2009native} is a sandbox for untrusted x86 code, specifically targeted at binary code to be executed in-browser. Native Client targets browser-based applications with computation that require native performance and is built from two layers of sandboxes: The inner sandbox performs binary analysis to ensure code is free from unsafe machine instructions and uses x86 segmented memory to restrict data access, while the outer sandbox performs kernel interposition to restrict access to syscalls.

\subsection*{2.5 Apiary}

\subsection*{2.6 MBOX}

\subsection*{2.7 System sandbox (for Microsoft Windows)}

\section*{3. Research Methodology}
We plan to evaluate a sample of widely-used process isolation and restricted computing environments ranging from lightweight \texttt{chroot}- and \texttt{seccomp}-based jails to full-blown virtual machines. The hard (quantitative and objective) metrics in our evaluation are: \textit{startup time}, \textit{execution speed}, \textit{syscall speed}, and \textit{memory overhead}. The soft (subjective) metrics are: \textit{hardware requirements}, \textit{software/operating systems requirements}, \textit{ease of setup and launch}, \textit{granularity of configuration}, and \textit{configuration management}. We expect that the soft metrics will provide some context for the quantitative differences between the isolation techniques and reveal tradeoffs between functionality and performance.
\newline\newline
Based on our literature review, we found these technologies and tools to be a representative sample of current methods of creating restricted computing environments:\vspace{0.5em}
{\small
\begin{itemize}
\item minijail and firejail
\item QEMU
\item Native Client
\item MBox
\item Linux Containers
\item FreeBSD jail
\item Docker
\item VirtualBox running Debian guest
\end{itemize}
}
\noindent
To collect hard metrics, we will run benchmarks that stress CPU and memory resources. We will then analyze the results and examine the implementations of each tool to explain measurement trends. This may be repeated in an iterative process, for example running the same benchmark on increasingly stricter security configurations for each restricted environment. As part of the measurement process, we will collect experience using these technologies to qualify our soft metrics.
\subsubsection*{3.1 Required Hardware}
Our tests will be performed on x86-64 Linux machines (which will require access to the CLIC lab, and maybe CRF accounts). We will use the same operating system distribution when possible. In the case of FreeBSD jail running on FreeBSD or minijail running on Chrome OS, for example, we will account for operating system differences in our analysis.
\subsubsection*{3.2 Required Software}
We seek to use the industry-standard SPEC CPU2017 benchmark \cite{limaye2018spec} as a test suite to gather performance metrics. However, we are not sure if it is practical, given its cost (\$1,000). Other possibilities include using open-source benchmarks such as PXZ and sysbench. These were used in a 2014 performance study on containers and virtual machines by IBM Research ~\cite{felter2014docker}. For measuring startup time and syscall time, we will use the \emph{perf} tool.
\section*{4. Hypothesis}
We hypothesize that there will be an inverse relationship between configurability and how ``light-weight'' a restricted computing tool is; that is, we expect that the more configurable or flexible the tool is to use (e.g. how granular of access controls can be set up), the longer it will take to start up and the larger memory footprint it will require. In particular, we anticipate Docker, VirtualBox VM, and QEMU to have significant startup overheads due to creation of a writable container layer for containers, and virtual disk creation for virtual machines. We expect execution time and memory overhead of containers and virtual machines to be close to native execution ~\cite{felter2014docker}. On the other hand, for sandboxes relying on syscall filtering methods like seccomp, we expect the main slowdown to be syscall speed due to interposition on a non-trivial fraction of those syscalls ~\cite{kim2013mbox}. While performance of containers and virtual machines on hard metrics have been studied, we hope to present measurements across a wider range of isolation techniques.
\section*{5. Projected Research Schedule}
\textbf{2/12-2/18} \\
\textbf{2/19-2/25}
\textbf{2/26-3/4}
\textbf{3/5-3/11}
\textbf{3/12-3/18} \\
\textbf{3/19-3/25}
\textbf{3/26-4/1}
\textbf{4/2-4/8}
\textbf{4/9-4/15}
\textbf{4/16-4/22}
\textbf{4/23-4/29}
\textbf{4/30-5/6}




\bibliographystyle{abbrv}
\bibliography{proposal}
\end{document}


